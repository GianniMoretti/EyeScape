\documentclass{article}
\usepackage[utf8]{inputenc}
\usepackage{graphicx}
\usepackage{verbatim}

\vspace{\stretch{1}}

\title{\LARGE Specifiche di progetto del sistema di controllo per acquari \\
\large Elaborato Progettazione e Produzione Multimediale}
\author{Gianni Moretti, Leonardo Pampaloni, Matteo Tinacci}
\date{A.A. 2021-2022}

\begin{document}

\maketitle

\begin{center}
\textbf{\large UNIVERSITA' DEGLI STUDI DI FIRENZE \\
Facolta di Ingegneria \\
\normalsize Corso di Laurea in Ingegneria Informatica}
\end{center}

\vspace{\stretch{1}}

\newpage

% Indice
\tableofcontents

\newpage

\section{Motivazione e Descrizione}
Il presente elaborato nasce dall'idea di uno di noi ragazzi che, lavorando in una pizzeria come cameriere, si \`e trovato ad interagire con un applicativo simile. Abbiamo cos\`i pensato di riprodurne uno personalizzato che preveda inoltre, considerato il periodo particolare che stiamo vivendo, la possibilit\`a di gestire azioni atte alla sicurezza dei clienti, come il loro monitoraggio per poterli rintracciare. \\
L'applicativo ha lo scopo di gestire le varie parti che compongono un ristorante, in modo da farle interagire e collaborare assieme. Abbiamo individuato le cinque figure professionali principali che possono trovarsi ad agire all'interno di un ristorante:

\begin{enumerate}
\item 
\end{enumerate}

\subsection{Possibili aggiunte}

\subsubsection{Ordini a domicilio o da asporto}

\subsubsection{Prenotazioni dei tavoli}  

\subsubsection{Login dei vari dipendenti}


\newpage

\end{document}